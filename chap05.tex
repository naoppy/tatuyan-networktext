\section{ベースバンド伝送方式}

\section{伝送速度}

%--- Tomohiro memo---
## 信号伝送方式
信号伝送方式にはベースバンド伝送方式と帯域伝送方式の2種類があります。

帯域伝送方式は搬送波にデータを乗せて伝送する方式で、振幅変調、位相変調、周波数変調、振幅位相変調といったものがこれの例に挙げられます。  
この中にある振幅位相変調は、二進数の桁を多くとることができるようになり、一度にたくさんのデータ(大きな値)を搬送波に載せることができるようになるといった利点があります。

ベースバンド伝送方式は0と1の電圧の差でデータを伝送する方式で、これにもいくつかの種類があります。 
例えば、

・return to 0方式(RZ方式)…一回に送るデータのうち半分を0にする方式。  
・non return to 0方式(NRZ方式)…上とは逆に、一回に送るデータの半分を1にする方式。  
・両極方式…ある+1という電圧から-1という電圧までを使う方式。仮に-1を1とすると、+1が0とされる。  
・単極方式…+1と0(または-1と0)の電圧を使う方式。

という4つの方式がありますが、これらは2つを組み合わせて使います。(ex.両極RZ方式、単極NRZ方式など)

ほかにも電圧の値を絶対値で取り、+1か-1かは前回取った方とは違う方という風に判定するバイポーラ方式、ビットの真ん中で電圧が変わったのを判定して、値が上昇したら1、下降したら0と判定するマンチェスター方式などがあります。

## 伝送速度
伝送速度は、一般的に「1秒間にどれだけのビットを送ることができるか」を数値で表したbps(bit per second)を用いて表します。




%--- Naoppy memo---

## 信号伝送方式

電気や光、電波など。

伝送方式にはベースバンド伝送方式と帯域伝送方式がある。

### 帯域伝送方式
搬送波にデータをのせて送る。

変調によって搬送波に載せる

直交振幅変調など...x、yの2つのパラメータを変えると、(xのパターン数×yのパターン)数の状態を表現できる。

###ベースバンド伝送方式
0と1の電圧で送る伝送方式。

return to zero方式...一度におくるものの半分は0で半分は1

not return to zero方式...ずっと1が送られる

両極方式...+1から-1までの電圧を使う、端が0と1

単極方式...+ or - 1から0までの電圧を使う、端が0と1

バイポーラー方式...+1と-1を使うが、+1と-1は両方1を表し、0は0を表す。前回1で1を表現したのなら、次は-1で1を表現しなければならない。

マンチェスター方式...あるbitを示す時間の間が-1 to 1(増えた)なら1 to -1(減った)なら0になる。
現代のインターネットはこれ


## 伝送速度
データ伝送速度...Mbpsなど

変調速度...どれだけのデータを1秒の搬送波に載せられるかを表す...単位はbaud


%--- memo ここまで---



\section{}

\section*{演習問題}
\begin{problems}
\item ほげ
\end{problems}
