電気を用いない通信から見てきた通り、通信と記録は密接な関係にある。また、記録の際には記録するべき要素の取捨選択が行われる。人間が直接運ぶにせよ光や音を使うにせよ電気通信にせよ、記録されたものを受け渡しするのが通信であるということに立ち戻れば、記録と通信は表裏一体の関係にあることは明らかであろう。

本章では、離散的な「デジタルデータ」をいかにして表現・記録するかという点について学んでいく。


%--- Tomohiro memo---
# 第5章 電気信号伝送の誤り制御

ここまでの章で、アナログのデータをデジタルデータにし、それを搬送波に乗せて送る、一連の流れを見てきました。この流れで確かにデータは送信できますが、伝送の最中に紛れうる誤差―混信・雑音・遮断―などが起きると厄介です。そこで、伝送時に誤りが入っていないか確かめるために、誤り制御という概念が生まれてきました。今回は、その誤り制御について学んでいきましょう。

## 誤り制御の概念
データの送信時に誤差によって誤ったデータが受信側に届くことがあります。この『データの誤り』を制御するのが誤り制御で、これをつける理由としては高確率できちんと伝送されているかを確かめる、といったことが挙げられます。

誤り制御に使うデータは大きかった場合意味がなくなります。なぜかというと、謝り制御に使うデータが大きいとそれ自体が誤っている可能性が非常に高くなるからです。

## 誤り制御方式
後述の誤り検出符号を使い、誤りが検出されれば再度データを送るように要求するものを帰還方式(ARQ - Automatic Repeat reQuest)と言います。

一方、これとは別の後述の誤り訂正符号を用いて自分で修正を行うような方式を無帰還方式(FEC - Forward Error Control)と言います。

## 様々な誤り制御符号
誤ったデータが送られて来た場合、それに付随して自分で誤りを直せるようなデータを送って受信側に直させるような符号を誤り訂正符号といいます。

一方、受信側では自分で治すために使うことができず、誤りがあるかないかだけがわかるデータを誤り検出符号と言います。

他にも、指定した範囲の1が偶数個なら最後に1を、偶数個なら0を付けるような形のパリティビット、データを全て足し、その数値のMODをとるというようなchecksum、ハミング距離を求めてxorを計算したものを制御符号とするハミング符号、元データの各ビットを係数化した多項式を生成し、決められた多項式で割った剰余をcheckbitとして誤りを判別するCRCなどがあります。

### 演習問題
今回紹介した各種の誤り制御符号を実装してみましょう。

## 授業の連絡
次回の授業は、ここまでの内容の復習の演習問題(理論問題)を行います。演習と解説の両方をするつもりですが、時間の制約が厳しいため、十分に回答できる時間がとれるかは保証の限りではありません。

このため、問題の内容を一読しておいてもらえると助かります。(もちろん、余力があれば前もって解いておいてもらえるとより良いです)。
%--- Ecasd memo---

# 第5章 電気信号伝送の誤り制御
ここまでの章で、アナログのデータをデジタルデータにし、それを搬送波に乗せて送る、一連の流れを見てきました。この流れで確かにデータは送信できますが、伝送の最中に紛れうる誤差―混信・雑音・遮断―などが起きると厄介です。そこで、伝送時に誤りが入っていないか確かめるために、誤り制御という概念が生まれてきました。今回は、その誤り制御について学んでいきましょう。

## 誤り制御の概念
データの誤りを判明させ,またその量を減らすことをいう.  
具体的には誤っているならば一致しない量を用意する.  
誤り制御の大きさは小さい方が好ましい,なぜならばそれ自身が誤っている可能性が存在するからだ.  

## 誤り制御方式
### 誤り検出符号 : 帰還方式
誤りの存在性だけを検出することをいう.  
例 ハッシュ, チェックサム, パリティビットなど  

### 誤り訂正符号 : 無帰還方式
以下に記す誤り検出符号を用いることで自己修正を行うような方式をいう.  
例 ハミング符号, 多重化, 縦横パリティなど  

## 様々な誤り制御符号
#### パリティビット : 検出
桁和に合わせて最後のビットを変更する.  

#### チェックサム : 検出
総和をとって MOD を取る.

#### ハミング符号 : 訂正
例えば 4bit のデータなら各符号は以下のようになる.  
```c1 = w1 ^ w3 ^ w4  
c2 = w1 ^ w2 ^ w4  
c3 = w1 ^ w2 ^ w3  
```

#### ハミング距離
2つの整数間のハミング距離は a xor b の立っている bit 個数で与えられる.  
```dist = __builtin_popcount(a ^ b);
```

### 演習問題
今回紹介した各種の誤り制御符号を実装してみましょう。


%--- memo ここまで---

\section{}

\section*{演習問題}
\begin{problems}
\item ほげ
\end{problems}
