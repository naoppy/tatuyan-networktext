通信とは、人の言を通ずることである。人の言とはすなわち信号であり、我々は具象をデータ化:信号化して記録してそれを通ずることにより通信としているのである。したがって、電気通信とは電気的信号を通ずることにほかならない。

では、それら電気的信号を伝送するにはどのようにすれば良いだろうか。最も単純には媒体の輸送という手法が挙げられる(これは現代で言うところのスニーカーネットにあたる)。だが、都度輸送するのは効率でも速度でも劣ることから、ご承知のとおり電気的なつながりを以って(人間が直接赴くこと無く)伝送を行うことが大半である。本章からは、電気信号を送る仕組みを学んでいく。

%--- Tomohiro memo---
# 第4章 電気信号の伝送

ここまでに、情報をどのようにして記録するかを見てきました。こうして符号化されたデータは2進数の0と1に対応するため、2元的な電気信号を生成し、これを送ることができれば電気通信となりそうです。もちろんその理解で良いのですが、単に電気信号を送るだけでは不便な部分も有ります。

この章では、電気信号そのものを送ることについて見ていきましょう。

いままでは0と1のデータがどこで始まってどこで終わるかがわかっていたから理解できた。しかし実際、その開始と終了は分からない。

こういう場合どうすればいい(どうやって同期すればいい)のか?

#電気信号の同期方式

##bit同期について

bit同期とは、理想的な電圧がかかっているとして、それをどこで区切って置き換えるか、一つの電気信号の0と1をどうやって定義するかを決めるものである。

##同期パルス方式

同期用の一定のパルスを決め(同期パルスという)、そのとおりに電気信号を送ることで一定区間内で0と1を解釈するという同期方式。

##調歩同期方式

電気信号の解釈は、極論最初の合図と最後の合図がわかれば一応することができる。
この最初の合図をスタートビット、終わりの合図をエンドビットといい、これを基準にして電気信号を解釈する同期方法が調歩同期方式である。

#データの区切りについて

電気信号自体の同期は出来ても、「関係のない信号とデータの境目」は意外とわからない。調歩同期方式だとデータの塊の判別はできるけどbit同期だとそれが出来ない。

ではデータの塊はどうすれば同期することができるのか?

##ブロック同期方式

データの塊も、どこから始まっているかという合図があれば同期することができる。
ブロック同期方式は、「データがどこからか」ということを解釈する合図を作り、それに合わせて同期するという同期方式である。

##フラグ同期方式

考えを変えてみると、ある特定の電気信号を決めて、それ「以外」をデータと解釈するという方法も考えられる。

その方法がフラグ同期方式で、決められたbit列をフラグパターンとし、そのフラグパターン「以外」をデータと解釈して取り出す同期方法である。

##キャラクタ同期方式

始めることを意味する特定の8文字の文字列を決め(これをSYN[シン]という)、そのSYNがきたらその後がデータであると解釈する同期方式。

###調歩同期方式との違いは?

調歩同期方式とキャラクタ同期方式の違いは、前者は「データ」を、後者は「電気信号」を解釈するという解釈するものの違いである。

### 演習問題:bit列のフラグ同期
bit列が与えられます。これは、伝送されてきたビット列とします。このとき、フラグパターンを01111110とするフラグ同期方式で同期を取り、そこから8bit毎に区切って得られるデータをchar型の文字列として出力するプログラムを作成してください。

ここまでで1回

---

ここからで1回


%--- memo ここまで---

\section{}

\section*{演習問題}
\begin{problems}
\item ほげ
\end{problems}
