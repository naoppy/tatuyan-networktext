気づけば、C言語の教科書を書いてから5年、問題集を書いてから2年が経とうとしている。"弟子"として1から教えた若人も手を離れ、弟子も他の知己も含め、時折尋ねに来た時に応ずるばかりとなった。今は、新たに学んだ日本茶でも淹れてほっと一息つきながら、最近はじめた落語に興じている日々である。

変わったのは何もプライベートばかりではない。時はすべてを連れて行くもので、仕事も住まいも知人も大きく変わった。それは、弟子や知己からの質問も例外ではない。ネットワークやシステムに関する質問が増えたのだ。仕事もまた然り。面白いことには、質問も仕事も、ネットワークは抽象化した理論をお話することが主で、システムは個々の具象を扱うことが多いというのが、共通していることである。

個々のシステムについては、各々をマニュアル的に記すことで満足な仕上がりが達せられる。DNS,LDAP,DHCP,POP,SMTPなど各々のサービスの設定は言うに及ばず、ルータの設定コマンドなども、マニュアルで対応すれば良いところであろう。だが、それらの設定の意味を理解することまでは、マニュアルに求められまい。殊に、汎論の範疇と呼ぶべき部分までマニュアルに記載しているようでは、あまりに冗長となってしまう。そこで、これらネットワークの汎論を統べる役目の一冊を用意したいと考えた。周辺の方への説明の際などに便利でもあろうし、無論、自身の記憶を掘り起こす良い復習の契機にもなろう…こうして、この本を著すこととした。

新たに記すときに悩むのは、その性格付けである。コンピュータネットワークの世界には、大部の聖典もあればよく纏まった入門書もある。同じ方向性で書いたとしても、陳腐なものにしかなりはしない。いかなる性格付けにするか、その悩みに鍵をくれたのは幸いに私を慕ってくれる優秀で賢明なる若き友であった。彼らの根問の中で、私は文字に象られている通りの「人に言を通す」という通信こそが大切であると考えたのである。口伝から始まり現代ではインターネットで多くの通信が行われる。その中で、私は非常に多くの通信方法を経験し、また利用している。趣味で演る落語は口伝が基本だ。生で見る落語は、肉声か、あるいはマイクを通す程度である。一方、レコードやCD、DVDで落語を見ることもある。手紙を丁寧に書いて送るのは実に楽しいもので、時折私書が郵便箱にあるとそれだけでとても嬉しい。一方、友人とのやり取りは電子メールやSNS、チャットアプリなどが主力となっている。これだけ多く「人の言を伝える」ことに接する現代、まずは口伝、それから手紙と俯瞰し、そこからアナログ通信・デジタル通信へと発展させていくのはどうだろうか、と考えた。

といっても、演劇の声の出し方や手紙の書きかたなどを丁寧に書くものではない。落語の稽古風景や郵便制度の話を書いたところで脱線にしかならない。そこで、これらの各分野の経験を元に、「人に言を通す」ことを掬い出して話す、そんなテキストを執筆することとした。歴史に通信を学び、そこから電気通信はどうなっていったのか、どういう必要性で技術が生まれたのか、経験から考察を加えて執筆していこうと思った。自身の学んだ電気通信を、経験に従って解釈しなおし、書き直す。地味であり、新規性も乏しいかも知れない。だが、何より話好きな自分の教え方と書くテキストに一貫性を入れるなら、言を通す、それを置いて他になかろう。オーソドックスかもしれないが、様々な分野の経験が何がしかのシナジーをおこし、説明に彩りを加えてくれればと思う。

本書の導入として、我々人類が物事を伝えるために用いた手段には如何なるものがあるのか、それは現代でどのように生きてあるいは滅びたのかという章を設けた。電気通信と程遠く見えるこれらの伝達には、しかし電気通信の要件を定めるような人類のアイディアが詰まっている。そのアイディアの最も元となる部分には「記録」があり、第I部ではその記録技術や信号処理技術についての最小限を記した。ここで記録された情報をまずは電気的に伝えることを考えるのが第II部である。ラジオ・電話・FAXと言った現代の「デジタル信号」からすると一代前の原理であるが、これを知ることはTCP/IP世界にも関わる低レイヤー技術の理解に十分に役立つと考える。第III部からはいよいよ本丸、コンピュータ・ネットワークへと論を進める。階層化モデルに従った理解と代表的なプロトコルを、各々第III部・第IV部として本書の大部分を用いて解説する。付章的であるが、第V部として、冗長化・セキュリティ・トラブルシューティングなど、構築や管理といったより実践的な情報の基盤となる、最初の一歩を記述した。

各章の記述においては、多少の数学を要求する箇所もあるが、高校程度の数学が取り扱えれば読めることを基本とした。ただ、特に前半の章においては厳密な取り扱いにどうしても数学が必要な部分がある。その部分については、まず厳密な取り扱いをせずに論を展開し、後の節で数学的議論を補うこととした。これらの部分には[補遺]と記しており、読み飛ばしても大局に影響しないよう心がけた。また、汎論に対する実例の紹介という意味付けで、各章で出来うる限り演習問題を出題し、巻末にその解答を記述した。汎論の解説ではあっても、手ずから具体例を解釈することは理解の深化に繋がると信ずる。

最後に、この本・これまでの本や、著者と付き合ってくれたこれまでの友人たち、付き合うこととなるこれからの読者たちにもお礼を申し上げたい。南天竺には赤栴檀という立派な木があり、その周には難莚草という草が蔓延ると聞く。この2つの植物は"有無相持ち"、赤栴檀の降ろす露が難莚草に双無き水となり、難莚草の盛衰は赤栴檀に栄養をもたらすという。友人と自身、読者と著者…携わる人々がそんな関係であることを祈り、この本を捧げる。

\begin{flushright}
桂米朝師の「伝」に畏敬を払いながら \\
達哉ん
\end{flushright}
